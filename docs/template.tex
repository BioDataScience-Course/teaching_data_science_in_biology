% !!!IMPORTANT NOTE: Please read carefully all information including those preceded by % sign
%Before you compile the tex file please download the class file AIMS.cls from the following URL link to the
%local folder where your tex file resides. http://aimsciences.org/journals/tex-sample/AIMS.cls.
\documentclass{aims}
\usepackage{amsmath}
  \usepackage{paralist}
  \usepackage{graphics} %% add this and next lines if pictures should be in esp format
  \usepackage{epsfig} %For pictures: screened artwork should be set up with an 85 or 100 line screen
\usepackage{graphicx}  \usepackage{epstopdf}%This is to transfer .eps figure to .pdf figure; please compile your paper using PDFLeTex or PDFTeXify.
 \usepackage[colorlinks=true]{hyperref}
   % Warning: when you first run your tex file, some errors might occur,
   % please just press enter key to end the compilation process, then it will be fine if you run your tex file again.
   % Note that it is highly recommended by AIMS to use this package.
   \hypersetup{urlcolor=blue, citecolor=red}
%\usepackage{hyperref}

  \textheight=8.2 true in
   \textwidth=5.0 true in
    \topmargin 30pt
     \setcounter{page}{1}

% The next 5 line will be entered by an editorial staff.
\def\currentvolume{X}
 \def\currentissue{X}
  \def\currentyear{200X}
   \def\currentmonth{XX}
    \def\ppages{X--XX}
     \def\DOI{10.3934/xx.xxxxxxx}

 % Please minimize the usage of "newtheorem", "newcommand", and use
 % equation numbers only situation when they provide essential convenience
 % Try to avoid defining your own macros

\newtheorem{theorem}{Theorem}[section]
\newtheorem{corollary}{Corollary}
\newtheorem*{main}{Main Theorem}
\newtheorem{lemma}[theorem]{Lemma}
\newtheorem{proposition}{Proposition}
\newtheorem{conjecture}{Conjecture}
\newtheorem*{problem}{Problem}
\theoremstyle{definition}
\newtheorem{definition}[theorem]{Definition}
\newtheorem{remark}{Remark}
\newtheorem*{notation}{Notation}
\newcommand{\ep}{\varepsilon}
\newcommand{\eps}[1]{{#1}_{\varepsilon}}

% This is for Pandoc Markdown -> LaTeX, but AIMS author guidelines do not allow
% custom LATeX macros except for equations!
%\providecommand{\tightlist}{%
%  \setlength{\itemsep}{0pt}\setlength{\parskip}{0pt}}
%
%  $if(csl-refs)$
%  \newlength{\cslhangindent}
%  \setlength{\cslhangindent}{1.5em}
%  \newlength{\csllabelwidth}
%  \setlength{\csllabelwidth}{3em}
%  \newenvironment{CSLReferences}[3] % #1 hanging-ident, #2 entry spacing
%   {% don't indent paragraphs
%    \setlength{\parindent}{0pt}
%    % turn on hanging indent if param 1 is 1
%    \ifodd #1 \everypar{\setlength{\hangindent}{\cslhangindent}}\ignorespaces\fi
%    % set entry spacing
%    \ifnum #2 > 0
%    \setlength{\parskip}{#2\baselineskip}
%    \fi
%   }%
%   {}
%  \usepackage{calc} % for \widthof, \maxof
%  \newcommand{\CSLBlock}[1]{#1\hfill\break}
%  \newcommand{\CSLLeftMargin}[1]{\parbox[t]{\maxof{\widthof{#1}}{\csllabelwidth}}{#1}}
%  \newcommand{\CSLRightInline}[1]{\parbox[t]{\linewidth}{#1}}
%  \newcommand{\CSLIndent}[1]{\hspace{\cslhangindent}#1}
%  $endif$



%% Place the running title of the paper with 40 letters or less in []
 %% and the full title of the paper in { }.
\title[$runtitle$] %Use the shortened version of the full title
      {$title$}

% Place all authors' names in [ ] shown as running head, Leave { } empty
% Please use `and' to connect the last two names if applicable
% Use FirstNameInitial.  MiddleNameInitial. LastName, or only last names of authors if there are too many authors
\author[Guyliann Engels, Philippe Grosjean and Frédérique Artus]{}

% It is required to enter 2020 MSC.
\subjclass{Primary: XXXXX; Secondary: YYYYY.}
% Please provide minimum  5 keywords.
 \keywords{Data Science Teaching, Flipped Classroom, R, RStudio, git.}

% Email address of each of all authors is required.
% You may list email addresses of all other authors, separately.
 \email{Guyliann.Engels@umons.ac.be}
 \email{Philippe.Grosjean@umons.ac.be}
 \email{Frederique.Artus@umons.ac.be}

% Put your short thanks below. For long thanks/acknowledgements,
%please go to the last acknowledgments section.
%\thanks{We would like to acknowledge...}

% Add corresponding author at the footnote of the first page if it is necessary.
% Please add <dollar>^*<dollar> adjacent to the corresponding author's name on the first page.
% The example shown in this template is if the first author is the corresponding author.
\thanks{\textsuperscript{*} Corresponding author: Guyliann Engels}

\begin{document}
\maketitle

% Enter the first author's name and address:
\centerline{\scshape Guyliann Engels\textsuperscript{*}, Philippe Grosjean}
\medskip
{\footnotesize
% please put the address of the first author
 \centerline{Numerical Ecology Department, Complexys and InforTech Institutes, University of Mons}
   %\centerline{Other lines}
   \centerline{Avenue du Champ de Mars, 8, 7000 Mons, Belgium}
} % Do not forget to end the {\footnotesize by the sign }

\medskip

\centerline{\scshape Frederique Artus}

\medskip
{\footnotesize
 % please put the address of the second  and third author
 \centerline{ Pedagogical Support and Quality Assurance Department, University of Mons}
  % \centerline{Other lines}
   \centerline{Place du Parc, 20, 700 Mons, Belgium}
}

\bigskip

% The name of the associate editor will be entered by an editorial staff
% "Communicated by the associate editor name" is not needed for special issue.
% \centerline{(Communicated by the associate editor name)}


%The abstract of your paper
\begin{abstract}
  $abstract$
\end{abstract}

$body$

% You may incorporate your references as follows in your main tex file.
% Using BibTex is not recommended but can be handled.

\begin{thebibliography}{99}
%$bibliography$

\bibitem{Alonso2019} [10.1016/j.compedu.2019.103612]
     \newblock  C. Alonso-Fernandez,  A. Calvo-Morata, M. Freire, I. Martínez-Ortiz and B. Fernandez-Manjon,
     \newblock Applications of data science to game learning analytics data: A systematic literature review,
     \newblock \emph{Computers \& Education}, \textbf{141} (2019), 103612.

\bibitem{Alvarenga2020} [10.1080/10691898.2020.1773354]
     \newblock  H. Alvarenga da Silva and A. Sampaio,
     \newblock Teaching introductory statistical classes in medical schools using RStudio and R statistical language: Evaluating technology acceptance and change in attitude toward statistics,
     \newblock \emph{Journal of Statistics Education}, \textbf{28} (2020), 212--219.

\bibitem{Auker2020} [10.1002/ecs2.3060]
     \newblock  L. A. Auker and E. L. Barthelmess,
     \newblock Teaching R in the undergraduate ecology classroom: approaches, lessons learned, and recommendations,
     \newblock \emph{Ecosphere}, \textbf{11} (2020), e03060.

\bibitem{Baker2016} [10.1038/533452a]
     \newblock  M. Baker,
     \newblock 1,500 scientists lift the lid on reproducibility,
     \newblock \emph{Nature}, \textbf{533} (2016), 452--454.

\bibitem{Banks2019} [10.1007/s10869-018-9547-8]
     \newblock  G. C. Banks, J. G. Field, F. L. Oswald, E. H. O'Boyle, R. S. Landis, D. E. Rupp, S. G. Rogelberg.
     \newblock Answers to 18 Questions About Open Science Practices,
     \newblock \emph{Journal of Business and Psychology}, \textbf{34} (2019), 257--270.

\bibitem{Baumer2014} [10.5070/t581020118]
     \newblock B. Baumer, M. Cetinkaya-Rundel, A. Bray, L. Loi and N. J. Horton,
     \newblock R Markdown: Integrating A Reproducible Analysis Tool into Introductory Statistics,
     \newblock \emph{Technology Innovations in Statistics Education}, \textbf{8} (2014).

\bibitem{Bernard2014} [10.1007/s12528-013-9077-3]
     \newblock  L. Lloyd-Smith,
     \newblock A meta-analysis of blended learning and technology use in higher education: from the general to the applied,
     \newblock \emph{J Comput High Educ}, \textbf{26} (2014), 87–-122.

\bibitem{Boettiger2015} [10.1145/2723872.2723882]
     \newblock  C. Boettiger,
     \newblock An Introduction to Docker for Reproducible Research,
     \newblock \emph{SIGOPS Oper. Syst. Rev.}, \textbf{49} (2015), 71--79.

\bibitem{Burton2011}
     \newblock  R. Burton, S. Borruat, B. Charlier, N. Coltice, N. Deschryver, F. Docq and al.,
     \newblock Vers une typologie des dispositifs hybrides de formation en enseignement supérieur,
     \newblock \emph{Distances et savoirs}, \textbf{9} (2011), 69--96.

\bibitem{Byers1989}
     \newblock  J.C. Byers, A. Bittner and S. Hill,
     \newblock Traditional and raw task load index (TLX) correlations: Are paired comparisons necessary?,
     \newblock \emph{Advances in Industrial Ergonomics and Safety}, 1989, 481--485.

\bibitem{Cetinkaya-Rundel2018} [10.1080/00031305.2017.1397549]
     \newblock  M. Cetinkaya-Rundel and C. Rundel,
     \newblock Infrastructure and Tools for Teaching Computing Throughout the Statistical Curriculum,
     \newblock \emph{American Statistician}, \textbf{72} (2018), 58--65.

\bibitem{Cetinkaya-Rundel2021} [10.1080/10691898.2020.1804497]
     \newblock  M. Cetinkaya-Rundel and V. Ellison,
     \newblock A Fresh Look at Introductory Data Science,
     \newblock \emph{Journal of Statistics Education}, \textbf{0} (2021), 1--27.

\bibitem{Cleveland2001} [10.2307/1403527]
     \newblock W.S. Cleveland,
     \newblock Data science: An action plan for expanding the technical areas of the field of statistics,
     \newblock \emph{International Statistical Review}, \textbf{1} (2001), 21--26.

\bibitem{Compeau2019} [10.1080/10691898.2020.1804497]
     \newblock P. Compeau,
     \newblock Establishing a computational biology flipped classroom,
     \newblock \emph{PLoS Computational Biology}, \textbf{15} (2019), 1--8.

\bibitem{Donoho2017} [10.1080/10618600.2017.1384734]
     \newblock  D. Donoho,
     \newblock 50 Years of Data Science,
     \newblock \emph{Journal of Computational and Graphical Statistics}, \textbf{26} (2017), 745--766.

\bibitem{Estrellado2020} [10.1007/978-1-4612-0873-0]
     \newblock R.A. Estrellado, E. A. Emily, J. Mostipak, J. M. Rosenberg and I. C. Velasquez,
     \newblock \emph{Data science in education using R},
     \newblock 1st edition, Routledge, London, England, 2020.

\bibitem{Fiksel2019} [10.1080/10691898.2019.1617089]
     \newblock J. Fiksel, L. R. Jager, J. S. Hardin and M. A. Taub
     \newblock Using GitHub Classroom to teach statistics,
     \newblock \emph{Journal of Statistics Education}, \textbf{27} (2019), 110--119.

\bibitem{Freeman2014} [10.1073/PNAS.1319030111]
     \newblock  D. R. Krathwohl,
     \newblock Active learning increases student performance in science, engineering, and mathematics,
     \newblock \emph{Proceedings of the National Academy of Sciences}, \textbf{111} (2014), 8410--8415.

\bibitem{Guzman2019} [10.1187/cbe.19-02-0041]
     \newblock L.M. Guzman, M.W. Pennell, E. Nikelski and D.S. Srivastava,
     \newblock Successful integration of data science in undergraduate biostatistics courses using cognitive load theory,
     \newblock \emph{CBE-Life Sciences Education}, \textbf{18} (2019), ar49 1--10.

\bibitem{Hart1988}
     \newblock S. G. Hart and L. E. Staveland,
     \newblock Development of NASA-TLX (Task Load Index): Results of Empirical and Theoretical Research,
     \newblock \emph{Advances in Psychology}, \textbf{52} (1998), 139--183.

\bibitem{Hsing2019} [10.1145/3287324.3287460]
     \newblock C. Hsing and V. Gennarelli,
     \newblock Using GitHub in the Classroom Predicts Student Learning Outcomes and Classroom Experiences: Findings from a Survey of Students and Teachers,
     \newblock In Proceedings of \emph{the 50th ACM Technical Symposium on Computer Science Education (SIGCSE '19)}, 2019, 672–678.

\bibitem{Kirschner2020} [10.4324/9780429061523]
     \newblock P.A. Kirschner and C. Hendrick,
     \newblock \emph{How learning happens: seminal works in educational psychology and what they mean in practice},
     \newblock 1st edition, Routledge, London, New York, 2020.

\bibitem{Kohonen1995} [10.1007/978-3-642-97610-0]
     \newblock T.  Kohonen,
     \newblock \emph{Self-Organizing Maps},
     \newblock 1st edition, Springer-Verlag, Berlin Heidelberg, 1995.

\bibitem{Krathwohl2002} [10.1207/s15430421tip4104]
     \newblock  D. R. Krathwohl,
     \newblock A Revision of Bloom' s Taxonomy: An overview,
     \newblock \emph{Theory Into Practice}, \textbf{41} (2002), 212--218.

\bibitem{Larwin2011} [10.1080/15391523.2011.10782572]
     \newblock  K. Larwin and D. Larwin,
     \newblock A meta-analysis examining the impact of computer-assisted instruction on postsecondary statistics education: 40 years of research,
     \newblock \emph{Journal of Research on Technology in Education}, \textbf{43} (2011), 253--278.

\bibitem{Lloyd-smith2010}
     \newblock  L. Lloyd-Smith,
     \newblock Exploring the Advantages of Blended Instruction at Community Colleges and Technical Schools,
     \newblock \emph{Journal of Online Learning and Teaching}, \textbf{6} (2010).

\bibitem{Martin2016}
     \newblock  F. Martin and A. Ndoye,
     \newblock Using Learning Analytics to Assess Student Learning in Online Courses,
     \newblock \emph{Journal of University Teaching \& Learning Practice}, \textbf{13} (2016), 69--96.

\bibitem{Marx2013} [10.1038/498255a]
     \newblock  V. Marx,
     \newblock The big challenges of big data,
     \newblock \emph{Nature}, \textbf{498} (2013), 255--260.

\bibitem{Onwuegbuzie2003} [10.1080/1356251032000052447]
     \newblock J. A. Onwuegbuzie  and V. A. Wilson,
     \newblock Statistics Anxiety: Nature, etiology, antecedents, effects, and treatments--A comprehensive review of the literature,
     \newblock \emph{Teaching in Higher Education}, \textbf{8} (2003), 195--209.

\bibitem{Romero2020} [10.1002/widm.1355]
     \newblock C. Romero and S. Vetura,
     \newblock Educational data mining and learning analytics: An updated survey,
     \newblock \emph{WIREs Data Mining Knowl Discov}, \textbf{10} (2020), e1355.

\bibitem{Rcoreteam2021}
     \newblock R Core Team (2021),
     \newblock R: A Language and Environment for Statistical Computing,
     \newblock R Foundation for
  Statistical Computing, Vienna, Austria. \url{
  https://www.R-project.org}

\bibitem{Rstudio2015}
     \newblock RStudio team (2015),
     \newblock RStudio Cloud,
     \newblock Boston, MA: RStudio, Inc.

\bibitem{Siemens2013} [10.1177/0002764213498851]
     \newblock G. Siemens,
     \newblock Learning Analytics: The Emergence of a Discipline,
     \newblock \emph{American Behavioral Scientist}, \textbf{57} (2013), 1380--1400.

\bibitem{Sousa2018} [10.1145/3287324.3287460]
     \newblock B. Sousa and D. Gomes,
     \newblock Teaching With R—-A Curse or a Blessing?,
     \newblock In Proceedings of \emph{the Tenth International Conference on Teaching Statistics (ICOTS10 '18)}, 2018, 672–678.

\bibitem{Spadafora2018} [10.5206/cjsotl-rcacea.2018.1.6]
     \newblock N. Spadafora and Z. Marini
     \newblock Self-Regulation and “Time Off”: Evaluations and Reflections on the Development of a Blended Course,
     \newblock \emph{The Canadian Journal for the Scholarship of Teaching and Learning}, \textbf{9} (2018).

\bibitem{Theobold2021} [10.1080/10691898.2020.1854636]
     \newblock A. S. Theobold, A. Hancock and S. Mannheimer,
     \newblock  Designing Data Science Workshops for Data-Intensive Environmental Science Research,
     \newblock \emph{Journal of Statistics and Data Science Education}, \textbf{29} (2021), S83--S94.

\bibitem{Wehrens2018} [10.18637/jss.v087.i07]
     \newblock R. Wehrens and J. Kruisselbrink,
     \newblock  Flexible Self-Organizing Maps in kohonen 3.0,
     \newblock \emph{Journal of Statistical Software, Articles}, \textbf{87} (2018), 1--18.

\bibitem{Wickham2019} [10.21105/joss.01686]
     \newblock H. Wickham et al,
     \newblock  Welcome to the Tidyverse,
     \newblock \emph{Journal of Open Source Software}, \textbf{43} (2019), 1686.

\end{thebibliography}

\medskip
% The data information below will be filled by AIMS editorial staff
Received xxxx 20xx; revised xxxx 20xx.
\medskip

\end{document}
