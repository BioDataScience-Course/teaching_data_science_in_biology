% Options for packages loaded elsewhere
\PassOptionsToPackage{unicode}{hyperref}
\PassOptionsToPackage{hyphens}{url}
%
\documentclass[
]{article}
\usepackage{amsmath,amssymb}
\usepackage{lmodern}
\usepackage{ifxetex,ifluatex}
\ifnum 0\ifxetex 1\fi\ifluatex 1\fi=0 % if pdftex
  \usepackage[T1]{fontenc}
  \usepackage[utf8]{inputenc}
  \usepackage{textcomp} % provide euro and other symbols
\else % if luatex or xetex
  \usepackage{unicode-math}
  \defaultfontfeatures{Scale=MatchLowercase}
  \defaultfontfeatures[\rmfamily]{Ligatures=TeX,Scale=1}
\fi
% Use upquote if available, for straight quotes in verbatim environments
\IfFileExists{upquote.sty}{\usepackage{upquote}}{}
\IfFileExists{microtype.sty}{% use microtype if available
  \usepackage[]{microtype}
  \UseMicrotypeSet[protrusion]{basicmath} % disable protrusion for tt fonts
}{}
\makeatletter
\@ifundefined{KOMAClassName}{% if non-KOMA class
  \IfFileExists{parskip.sty}{%
    \usepackage{parskip}
  }{% else
    \setlength{\parindent}{0pt}
    \setlength{\parskip}{6pt plus 2pt minus 1pt}}
}{% if KOMA class
  \KOMAoptions{parskip=half}}
\makeatother
\usepackage{xcolor}
\IfFileExists{xurl.sty}{\usepackage{xurl}}{} % add URL line breaks if available
\IfFileExists{bookmark.sty}{\usepackage{bookmark}}{\usepackage{hyperref}}
\hypersetup{
  pdftitle={Teaching Data Science to Students in Biology using R, RStudio and Learnr: Analysis of Three years Data},
  hidelinks,
  pdfcreator={LaTeX via pandoc}}
\urlstyle{same} % disable monospaced font for URLs
\usepackage[margin=1in]{geometry}
\usepackage{longtable,booktabs,array}
\usepackage{calc} % for calculating minipage widths
% Correct order of tables after \paragraph or \subparagraph
\usepackage{etoolbox}
\makeatletter
\patchcmd\longtable{\par}{\if@noskipsec\mbox{}\fi\par}{}{}
\makeatother
% Allow footnotes in longtable head/foot
\IfFileExists{footnotehyper.sty}{\usepackage{footnotehyper}}{\usepackage{footnote}}
\makesavenoteenv{longtable}
\usepackage{graphicx}
\makeatletter
\def\maxwidth{\ifdim\Gin@nat@width>\linewidth\linewidth\else\Gin@nat@width\fi}
\def\maxheight{\ifdim\Gin@nat@height>\textheight\textheight\else\Gin@nat@height\fi}
\makeatother
% Scale images if necessary, so that they will not overflow the page
% margins by default, and it is still possible to overwrite the defaults
% using explicit options in \includegraphics[width, height, ...]{}
\setkeys{Gin}{width=\maxwidth,height=\maxheight,keepaspectratio}
% Set default figure placement to htbp
\makeatletter
\def\fps@figure{htbp}
\makeatother
\setlength{\emergencystretch}{3em} % prevent overfull lines
\providecommand{\tightlist}{%
  \setlength{\itemsep}{0pt}\setlength{\parskip}{0pt}}
\setcounter{secnumdepth}{5}
\ifluatex
  \usepackage{selnolig}  % disable illegal ligatures
\fi
\newlength{\cslhangindent}
\setlength{\cslhangindent}{1.5em}
\newlength{\csllabelwidth}
\setlength{\csllabelwidth}{3em}
\newenvironment{CSLReferences}[2] % #1 hanging-ident, #2 entry spacing
 {% don't indent paragraphs
  \setlength{\parindent}{0pt}
  % turn on hanging indent if param 1 is 1
  \ifodd #1 \everypar{\setlength{\hangindent}{\cslhangindent}}\ignorespaces\fi
  % set entry spacing
  \ifnum #2 > 0
  \setlength{\parskip}{#2\baselineskip}
  \fi
 }%
 {}
\usepackage{calc}
\newcommand{\CSLBlock}[1]{#1\hfill\break}
\newcommand{\CSLLeftMargin}[1]{\parbox[t]{\csllabelwidth}{#1}}
\newcommand{\CSLRightInline}[1]{\parbox[t]{\linewidth - \csllabelwidth}{#1}\break}
\newcommand{\CSLIndent}[1]{\hspace{\cslhangindent}#1}

\title{Teaching Data Science to Students in Biology using R, RStudio and
Learnr: Analysis of Three years Data}
\author{}
\date{\vspace{-2.5em}}

\begin{document}
\maketitle

\hypertarget{abstract}{%
\section{Abstract}\label{abstract}}

\textbf{This is the original abstract that should be reworked according
to final content of the manuscript.}

The courses in biostatistics in biology at the University of Mons,
Belgium, were completely refactored in 2018 into data science courses
(see \url{http://bds.sciviews.org}). The content is expanded beyond
statistics to include computing tools, version management, reproducible
analyses, critical thinking and open data. Flipped classroom approach is
used. Students learn with the online material and they apply the
concepts on individual and group projects using a preconfigured virtual
machine with R and RStudio. Activities (H5P, learnr or Shiny
applications) are recorded in a MongoDB database (300,000+ events for
180+ students and 2,000+ GitHub repositories at
\url{https://github.com/BioDataScience-Course}). The analysis of these
data reveals several trends. (1) There is a relatively long lag period
required for the students to get used to the computing environment, the
teaching method and the data science in general. (2) Implication is very
high, with more than 85\% of the students that complete all the
activities and got good to excellent assessment. (3) There is a gap
between students' own perception of their skills achievements and their
assessment results: they tend to underestimate their progress. (4)
During COVID-19 pandemic lockdown, the intensity of the activities
largely decreased during two weeks before returning to previous level,
but for 3/4 of the students only. The remaining fraction never caught
up. We hypothesize that the technical requirements or the lack of
motivation during the lockdown were detrimental to roughly one student
over ten, despite all the efforts the University deployed to reduce the
social fracture.

\hypertarget{introduction}{%
\section{Introduction}\label{introduction}}

In a context where there is an exponentially growing mass of data (Marx
2013), a reproducibility crisis in Science (Baker 2016), and a
progressive adoption of Open Science practices (Banks et al. 2019),
statistics were broaden to a larger discipline called Data Science. For
the Data Science association, ``the Data Science means the scientific
study of the creation, validation and transformation of data to create
meaning'' (\url{http://www.datascienceassn.org/code-of-conduct.html}).
These changes also led to the emergence of data science programs in
universities and higher schools (Donoho 2017; Çetinkaya-Rundel and
Ellison 2021). One example is the Harvard Data Science initiative
(\url{https://datascience.harvard.edu/about}) initiated in 2017. With a
broader approach, comes also a broaden public. The data science courses
are not just limited to computer scientists, mathematicians or
statisticians, but also welcome students in humanities, social sciences,
and natural sciences (for instance, the data science training at Duke
University (Çetinkaya-Rundel and Ellison 2021)). Main focus of such
courses is for students to develop the ability to deal with ``real''
datasets in all their complexities and to realize reproducible analyses
to interpret these data in the light of knowledge in their field of
expertise.

The data transformation part of the job is a challenge for students with
a poor or no background at all in computing. Students that are not used
to deal with computer languages enter in a foreign world and have to
deal with many exotic concepts, techniques and tools. This is the same
for the analysis of these data when students have no background in
mathematics or statistics. It generates anxiety (see for instance
(Onwuegbuzie and Wilson 2003), for students in biology). The course must
be organized in a way that such students progress by little steps in
order to avoid exposition to much intimidating concepts and tools at
once. Hence, a student in computing science already masters one or more
computing languages, is acquainted with version control systems, with
databases and with the way data are represented in a computer. A student
in mathematics or statistics is familiar with various concept that
underpin the techniques to analyse the data. On the other hand, students
in biology, medicine, psychology, social sciences, economics, \ldots{}
have very different \emph{a priori} knowledges. Version control systems
like git, and their internet hosting counterparts like GitHub, Gitlab or
Bitbucket also make part of the tools that data science course teach and
use (Fiksel et al. 2019; Hsing and Gennarelli 2019). Presentation of the
results and the use of documents formats that dissociate content from
presentation, namely LaTeX, Jupyter Notebook, or R Markdown also
contribute to the large number of potentially new tools students have to
learn (Baumer et al. 2014).

Suitable computer hardware and software environments are required in the
practical sessions of the courses. Different approaches range from
inline software (RStudio Cloud (\url{https://rstudio.cloud/}),
Chromebook data science
(\url{http://jhudatascience.org/chromebookdatascience/})) to local
installation on the Student's computers. The former requires an
infrastructure to run the software on a server, nad that software is
only accessible to the students during the course. The later raises
problems of license for proprietary software, but also installation and
configuration issues. An intermediary solution uses preconfigured
virtual machines, or containers (e.g., Docker) (Çetinkaya-Rundel and
Rundel 2018 ; Boettiger 2015). Such a solution is the most flexible
because it can be deployed almost anywhere (in the computer lab, at
home, using a laptop, \ldots). To fix theoretical concepts through
applied exercises is a key aspect of learning data science (Larwin and
Larwin 2011). Correct choice of software is critical and exposing
students early with the tools they are most susceptible to use later in
their work is desirable. This was highlighted by (Auker and Barthelmess
2020) for instance, for the analysis of ecological data.

These data science courses pose several challenges to pedagogy because
various, numerous and unfamiliar concepts must be acquired by a
population of potentially very diverse students. Learning objectives
span a large range of cognitive abilities (Krathwohl 2002). {[}We need
to develop here things like flipped classroom, continuous evaluation,
pedagogy by projects, and inclusive pedagogy{]}. The flipped classroom
approach allows students to be active in their learning, which has the
benefit of improving student outcomes (Freeman et al. 2014).

{[}Partie pédagogie à détailler un peu, probablement sur 2 ou 3
paragraphes{]}

Recently, data science is also used to analyze the effect of different
pedagogical practices on the outcome of these courses {[}Estrellado et
al. (2020); second ref to add{]}. A vast amount of data can be collected
on students activities, and the analysis of these data allows to compare
the impact of different pedagogical approaches, or to quantify and
document the impact of changes in the courses.

At the University of Mons in Belgium, we have started to rework our
biostatistics courses in the biology curriculum in 2018. A series of
Data Science courses were introduced, both for our undergraduate and
graduate students. These courses are inspired from precursor initiatives
cited here above. The goal of these courses is to form biological data
scientists capable to extract meaningful information from raw biological
data, and to do so in a reproducible way, with correct application of
statistical tools and an adequate critical mindset. A preconfigured
VirtualBox virtual machine with R, RStudio, Rmarkdown, git, and a series
of R packages preinstalled is used (url sciviews box?) as a very
flexible way to deploy the same software environment both on the
university computers and on student's own laptops.

As our course were completely reworked, we also decided to use flipped
classroom and progressive adoption of suitable pedagogical practices
with a cyclical approach that consists in stating goals, building
pedagogical material with a large emphasis on numerical tools and
collection of student's activities, and finally, analysis of the data
collected. Conclusions of these analyses initiate another cycle the
following academic year with refined goals and pedagogical materials and
techniques. Here, we present the main results spanning on three
successive academic years from 2018 to 2021, including two particular
periods where distance learning was forced due to COVID pandemic
lockdown.

{[}TODO: present here the 3-4 research questions that will be elaborated
in the manuscript.{]}

\begin{itemize}
\item
  examen final versus évaluation de projet
\item
  profils d'étudiants.
\item
  timing et support présentiel - distanciel.
\item
  charge cognitive learnr
\end{itemize}

\hypertarget{methods}{%
\section{Methods}\label{methods}}

The course materials are available online
(\url{https://wp.sciviews.org}) and are centralized in a Wordpress site.
Students have to login with their GitHub account and their academic data
are collected from the UMONS Moodle server. The courses are break down
into modules that amount roughly to 15h of work each in total. There are
two sessions of 2h and 4h in the classroom (outside of lockdown periods,
of course). Main activities in the class are actual data analysis
(projects), answering student questions, and very sort lectures of 1/4h
on selected topics. Students propose and vote for the topics to be
covered during these short lectures. Finally, we encourage students to
help each other and to explain what they understand to their colleagues.
Indeed, students' questions may be redirected to other students that
have already mastered the topic by the educators. On the other hand,
teachers rarely answer questions directly. When it is possible, they
rather propose new tracks or ideas to investigate in order to find the
solution by oneself. Students that have finished the work before the
others are encouraged to help their colleagues too.

Regarding the timing, one module it taught every second week so that
students have enough time to prepare the material at home and then, to
finalize their projects before the next module. Since a term is 14
weeks, we do not teach more than six modules in a course unit to avoid
compacting them in time at a faster pace than one module every second
week.

All student activities in H5P exercises for their auto-evaluation, and
in learnr tutorials to transition smoothly from the theory to the
practice are recorded in the MongoDB database. The learnitdown R package
(\url{https://www.sciviews.org/learnitdown/}) provides the functions
required to manage user login, user identification and activity tracking
in the interactive material.

Projects with the data, the analyses and te reports are hosted in GitHub
repositories. These repositories are cloned and edited locally with
RStudio, either on a PC in the computer lab, or directly on the
student's laptop. We encourage our students to install the virtual
machine on their own computer so that they can use it for other courses
too. Assignment and creation of the GitHub repositories for each
student, or group of students is orchestrated with GitHub Classroom. All
repositories are ultimately cloned in a centralized area on our servers
and data about commits (git logs) are collected using git version
{[}XXX{]} and R version 4.0.5. To give an idea of the amount of data
recorded, in 2020-2021 we have a little bit more than 2,500 events per
student.

In distance learning, support to the students was done via email and
Discord. At the end, all messages that were exchanged are collected
together into text files. These files are scraped using R code to create
a table with key information (basically, who, when, and what) for each
message. Surveys are periodically conducted during lessons by means of
Wooclap questionnaires (see, for instance, the Nasa-LTX questionnaire
analysis in the results section). Wooclap allows to export data into
Excel files. These data are then converted into a table in our database.

Information about users, courses, lessons and projects, as well as
grading items (on average, more that 130 grading items were established
for each student in 2020-2021) are anonymized: name, email and all the
personal information are replaced by random identifiers. The different
tables are ultimately exported into CSV files and made public. These
data are available at {[}\ldots{} Zenodo?{]}. Data collection,
treatment, and use respect European GDPR (General Data Protection
Regulation) since each student had to agree explicitly with the way data
are collected and used (including for research purpose) before the
course begins. They can visualize their own data through personalized
reports at anytime.

The course material is organized in a way that favour autonomy and
auto-evaluation (direct feedback in the exercises, hints and retry
button in case of wrong answer). Activities span into a sequence of
exercises of increasing difficulty, ranging from Level 1 to level 4.
Table 1 summarizes main characteristics of the exercises according to
the level.

\begin{longtable}[]{@{}
  >{\raggedright\arraybackslash}p{(\columnwidth - 4\tabcolsep) * \real{0.06}}
  >{\raggedright\arraybackslash}p{(\columnwidth - 4\tabcolsep) * \real{0.78}}
  >{\raggedright\arraybackslash}p{(\columnwidth - 4\tabcolsep) * \real{0.16}}@{}}
\caption{for levels of increasing difficulties in the
exercises.}\tabularnewline
\toprule
Level & Description & Type \\
\midrule
\endfirsthead
\toprule
Level & Description & Type \\
\midrule
\endhead
L1 & Short exercise directly integrated in the course and with direct
feedback for auto-evaluation & h5p \\
L2 & Guided exercise with contextual feedback within a short tutorial &
learnr \\
L3 & Individual and guided data analysis & individual project \\
L4 & More complex and free data analysis and reporting (group of 2 or 4
students) & group project \\
\bottomrule
\end{longtable}

{[}one or two paragraphs to describe statistical methods used
here\ldots{]}

The NASA-LTX questionnaire is composed of six questions on a Likert
scale to quantify the perceived workload to complete a task (Hart and
Staveland 1988). The questions concern mental load, physical load, time
pressure, expected success, effort required, and frustration experienced
during the accomplishment of the task. The average value for the six
questions constitutes a Raw Task Load indeX (RTLX) (Byers, Bittner, and
Hill 1989) that we use to quantify how students perceive the workload of
a given task.

\hypertarget{results}{%
\section{Results}\label{results}}

In all our three courses in biological data science, practice is the
most important activity. Our goal is to ensure that our students are
able to analyse all kind of real datasets, using the right techniques
and with a critical mind. They also learn how to write these analyses by
using R and R Markdown to create reproducible reports managed under
version control (git). There are several critical stages:

\begin{itemize}
\item
  Once they have learnt the principles in the book and auto-evaluated
  their comprehension of the concepts using H5P exercises (level 1
  difficulty), they have to get used to the software environment. Learnr
  tutorials (level 2) are used to gently introduce them to the R code
  required for the analyses by guiding them through their first data
  analysis. These tutorials are thus the entry point for the practice.
  We assess here the perceived workload of these tutorials to make sure
  they engage the students without exhausting them.
\item
  Projects, but individual (level 3) and in group of 2 to 4 students
  (level 4) represent the core activity. Evaluation of these projects
  constitute, thus, the core information to assess the competences of
  our students. However, an exam at the end of the course is a common
  practice. So, we compare grading our students obtain from such an exam
  with score they obtain directly in their projects. The final exam is
  written in learnr, and it mixes questions about the theory with partly
  solved data analyses they have to explain, criticize and continue
  during the exam session.
\item
  Despite we have relatively homogeneous classes of students with
  similarly (low) level of knowledge for statistics and computing at the
  beginning, the flipped class approach and the proactive attitude they
  have to develop (they must formulate their questions whenever they
  face a problem), we observe they develop very different strategies.
  Not all students ask questions. Some of them try to find solutions on
  their own. Some other prefer to ask their questions in private, while
  others have no problems to expose their difficulties on a public forum
  (a Discord channel for the course). The way and the timing they
  progress in the exercises also largely varies. The schedule is not
  tight and only suggest the rhythm of progression. No student is
  penalized if the exercises are done later, as soon as they are
  completed before the final deadline. Some strategies are more
  efficient than others. We analyse traces from the student activities
  to separate the different profile and we correlated them with the
  grade they obtain at the end of the course.
\item
  Finally, lockdown was imposed relatively abruptly and may interfere
  with the learning strategies they have developed. We analyse whether
  the switch to face-to-face activities to distance teaching and back
  has an impact on their productivity.
\end{itemize}

This study is performed all along the three courses that comprise 26
modules in total in 2020-2021. Table XXX summarizes the number of H5P,
learnr, individual and group GitHub projects that students have to
complete. It should be noted that for course C, we also introduced a
challenge in machine learning that replaced one group GitHub project.

\begin{longtable}[]{@{}lrrrlrr@{}}
\caption{Number of students, modules, and exercises for each course. For
the learnr tutorials, the first number is the amount of tutorial
documents and the second number in parentheses is the total number of
questions in these tutorials.}\tabularnewline
\toprule
Course & Students & Modules & H5P & Learnr & Indiv. projects & Group
projects \\
\midrule
\endfirsthead
\toprule
Course & Students & Modules & H5P & Learnr & Indiv. projects & Group
projects \\
\midrule
\endhead
A & 59 & 12 & 59 & 24 (211) & 10 & 4 \\
B & 45 & 8 & 29 & 11 (108) & 12 & 2 \\
C & 26 & 6 & 19 & 7 (37) & 7 & 1 \\
\bottomrule
\end{longtable}

\hypertarget{perceived-cognitive-workload-in-learnr-tutorials}{%
\subsection{Perceived cognitive workload in learnr
tutorials}\label{perceived-cognitive-workload-in-learnr-tutorials}}

In our courses, learnr tutorials play an essential role in the
progressive acquisition of competences because they are at the
transition between the theory (online book chapters) and the practice
(projects where student analyse real biological data by themselves). Our
goal is to prepare our students optimally for the practice of data
science. In the other hand, we don't want to exhaust their mental energy
in these tutorials before they start their projects. The efficiency of
these tutorials is qualitatively determined by observing the behaviour
of the students when they start their practical work.

A few tutorials were elaborated during the academic year 2018-2019, and
positive feedback on their utility (both by direct observation of the
pupils, and by their remarks) led us to systematize them into what we
now call level 2 activities (see Table XX) in the form of learnr
documents in 2019-2020. The tutorials were further refined in 2020-2021:
we added contextual hints thanks to the gradethis R package. When
students submit their answer to the exercises, the R code is analysed
and the results are compared with the solution. In case of differences,
heuristics are used to provide contextual hints. Students can then
refine their solution and resubmit it. This appears very efficient in
self-teaching and self-evaluation of their competences before switching
to the practice. However, the cognitive load required to perform these
exercises has, as far as we know, not been studied yet. We used a NASA
LTX questionnaire to assess it across all three courses. Participation
was 48/59 (81\%), 35/45 (78\%) and 18/26 (69\%) for courses A, B, and C
respectively.

\begin{figure}
\centering
\includegraphics{teaching_data_science_files/figure-latex/rtlx-1.pdf}
\caption{Perceived workload for the learnr tutorials in the three
courses. The black circle is the mean RTLX value. The number above each
box is the number of respondants.}
\end{figure}

The difficulty of the course, and thus, of the exercises in the
tutorials increase from one course to the other. However, we do not
observe an increase in the RTLX index. On the contrary, it is
significantly lower for course C than for course A (Tukey HSD, p-value =
0.023) TODO: INCOMPLETE REFERENCE FOR THE TEST (MUST INDICATE ANOVA
RESULTS FIRST !). The cognitive load perceived by the students
diminishes. This may be a consequence of a more fluent R coding and the
better mastering of the software environment.

\begin{verbatim}
## Analysis of Variance Table
## 
## Response: rtlx
##           Df  Sum Sq Mean Sq F value  Pr(>F)  
## course     2   926.9  463.47  3.5883 0.03134 *
## Residuals 98 12658.0  129.16                  
## ---
## Signif. codes:  0 '***' 0.001 '**' 0.01 '*' 0.05 '.' 0.1 ' ' 1
\end{verbatim}

\begin{verbatim}
## 
##   Simultaneous Tests for General Linear Hypotheses
## 
## Multiple Comparisons of Means: Tukey Contrasts
## 
## 
## Fit: lm(formula = rtlx ~ course, data = workload_rtlx)
## 
## Linear Hypotheses:
##            Estimate Std. Error t value Pr(>|t|)  
## A - B == 0    2.356      2.526   0.933   0.6183  
## C - B == 0   -6.058      3.296  -1.838   0.1607  
## C - A == 0   -8.414      3.141  -2.679   0.0229 *
## ---
## Signif. codes:  0 '***' 0.001 '**' 0.01 '*' 0.05 '.' 0.1 ' ' 1
## (Adjusted p values reported -- single-step method)
\end{verbatim}

\hypertarget{final-exam-versus-project}{%
\subsection{\texorpdfstring{Final exam \emph{versus}
project}{Final exam versus project}}\label{final-exam-versus-project}}

In 2018-2019 and 2019-2020, the evaluation was based on the completion
of a project and on a more conventional examination at the end of the
term. {[}describe a little bit the questions in the exams as being
focused on the practice{]}.

\begin{figure}
\centering
\includegraphics{teaching_data_science_files/figure-latex/exams_projects-1.pdf}
\caption{Grade (/10) obtained at the final exam in function of grade
obtained for the projects for courses A and B (two successive years for
A).}
\end{figure}

The comparison of the marks obtained by each student for a project and a
final exam shows only a weak correlation between these two types of
evaluations. Year 2018-2019 marks the transition to our flipped
classroom approach in teaching data science. Only one student failed in
the project, while almost one third of the students failed their final
exams. In 2019-2020, we raised a little bit the difficulty for the
project, resulting in a more widespread distribution of the results, but
with a similar pattern showing very little correlation between the two
evaluation methods. The same conclusion can be drawn for course B.

Despite a final examination that includes a series of practical
questions (writing R code to analyse data, as in the projects), this
type of assessment does not reflects the ability of the students to
correctly process and analyse biological data. Following these results,
the final examination is abandoned for the year 2020-2021, and it is
replaced by a continuous evaluation of the students activity across all
four level exercises (H5P, learnr, individual and group projects). These
activities are analysed in the following section.

\hypertarget{students-profiles}{%
\subsection{Students profiles}\label{students-profiles}}

The course material is enriched and organized in the four successive
difficulty levels presented in Table XXX. At all levels, the activity of
the students in the level 1 (H5P) and 2 (learnr) exercises is recorded
in a database. For the projects (levels 3 and 4), it is the git log data
that are used. During lockdown periods, exchange with the students and
answers to their questions were exclusively done by email, text or voice
exchanges on Discord, either on private or public channels. Students
were allowed to freely chose their favourite way to interact with the
teachers, or with each other. All these exchanges were recorded too.
Finally, all activities were used to establish the final grade for the
course, with a varying weight (for instance, course A second term, the
next weights were applied: 5\% for level 1, 10\% for level 2, 35\% for
level 3 and 40\% for level 4). On average, each student received more
than 130 assessments that accounted for the final grade. Two third of
these assessments were set manually by examining their projects and
using evaluation grids. The other third are scores automatically
calculated on the various exercises.

For the three courses, we recorded a total of more than 450,000 events,
which makes on average, almost 3,500 events for each student. The data
is summarize into sixteen metrics to quantify different aspects of their
learning strategies:

\begin{itemize}
\tightlist
\item
  questions: the number of questions they asked in total, divided by the
  number of modules in the course,
\item
  q\_mail: the number of emails send, divided by the number of modules,
\item
  q\_pub\%: the fraction of questions that were posted into public areas
  (\%),
\item
  q\_prod: the relative ``productivity'' of their questions as
  quantified by the number of lines changed in their reports per message
  sent,
\item
  h\_ok\%: the fraction of H5P exercises that were correctly answered,
\item
  h\_trials: the average number of trials for each H5P exercise
  (students can retry as often as they want),
\item
  l\_ok\%: the fraction of learnr exercises that were completed with a
  correct answer,
\item
  l\_trials: the average number of trials for each learnr exercise,
\item
  l\_hints: in learnr exercises, students can display hints to help them
  to solve the problems (but they lose 10\% of the score of the exercise
  for each hint). This is the average number of hints per exercise that
  were displayed,
\item
  i\_commits: the number of commits in individual projects divided by
  the number of individual projects,
\item
  i\_changes: the number of lines added or deleted in the reports of
  individuals projects, divided by the number of individual projects,
\item
  g\_commits: same as i\_commits, but for group projects,
\item
  g\_changes: same as i\_changes, but for group projects
\item
  g\_contrib\%: contribution of the student to the total work in group
  projects (in \% of all line changes done),
\item
  done\%: the fraction of all exercises done by the student,
\item
  intime\%: the fraction of the completed exercises that were done in
  time in \% (respect of the proposed schedule).
\end{itemize}

{[}supplementary data: distribution of the metrics and correlation
between them{]}

A Kohonen's self-organizing map was used to create student profiles
according to their activity, see Fig. XXX. A 3x3 hexagonal cells pattern
was chosen, and students are thus classified into nine different groups.

\begin{figure}
\includegraphics[width=6.06in]{figures/som} \caption{Self-organizing map of the student activities across the three courses in 2020-2021.}\label{fig:som}
\end{figure}

The small plots in gray scales show how metrics distribute in the nine
cells, from lowest value in white to highest value in black. They allow
to understand the way students behave according to their profile. Dots
in the central plot are the various students, with colour representing
the course (red = An, purple = B, blue = C) and the diameter of the dots
representing the grade the students obtained at the end of the course.

\begin{itemize}
\item
  Cell 3 contains students that completed a very small number of
  exercises (D), all the other students have completed most if not all
  exercises. These students obtained very low grades, of course. They
  belong to courses A and B.
\item
  At the opposite, cell 4 contains a large majority of the students of
  all three courses. These students have done a good portion of their
  exercises in time (A), have participated actively in public
  discussions (E) and have contributed on an average way to projects (H,
  and I). This is the most common pattern, which leads to success (good
  to excellent grades)
\item
  Cell 1 contrasts with cell 4 because exercises are completed almost
  always in time (A), the number of trials per learnr exercise if the
  highest (B, also for H5P exercises, not shown), but the participation
  to public discussions (E), and even the total number of questions (G)
  are lower than for cell 4. These students belong almost all to course
  A. Their pattern corresponds to more ``scholar'' students that try to
  solve the exercises by testing (sometimes randomly) various possible
  answers. They think less and try less to understand what they do. They
  follow instructions more closely, and consequently, they finish the
  exercises on time more often that all the others. Although being
  successful, we try to avoid this pattern since we want them to think
  about what they do, and to ask questions when they are stuck. It seems
  they change their behaviour for courses B and C towards cell 4.
\item
  Cell 7 contains a few students that obtained excellent grades, mostly
  from course C. These students are very active in public discussions
  (E) and they ask a lot of questions (G). They contribute also a lot to
  project reports (I). They also show among the lowest trial per learnr
  exercise (B) and they never used the hints (C). This pattern is
  typical for students that are very fluent with R code, that understand
  and apply heavily the concepts, and are curious about everything in
  the course. This is the typical positive behaviour we would like to
  observe as much as possible. Note the dynamic from course to course
  from cell 1 (``scholar'' students), to cell 4 (``typical'' behaviour),
  and cell 7 (``active and efficient'' students).
\item
  Cells 5 and 8 are mainly characterized by an usually large
  contribution to group projects (I). We noted this pattern is mostly
  due to a compensatory behaviour: their colleagues do not make their
  part of the job, and they take thus a larger part. This denotes groups
  that do not work well together. Cell 5 correspond to students that are
  more active on public channels than for cell 8 (E), but for the rest,
  they exhibit an average pattern.
\item
  Cells 6 and 9 presents students that seems to have more difficulties
  to communicate with the teachers: the number of questions is low (G),
  and never or rarely through public channels (E). Cell 6 contains
  students trhat mainly rely to email for their echanges, with a mild
  activity in the projects (H, I), but done mainly later (A). Students
  in cell 9 on the other hand, do not communicate at all (E, F, G).
  Consequen tly, their production appears artificially high per message
  (J), but it is poor on average (H \& I). These student have a large
  range of grades: some of them seem to work well in autonomy, while
  others obtain less good results. Anyway, there are certainly
  communication problems in distance learning with these students.
\item
  Finally, cell 2 contain students that manifest difficulties in a
  totally different way: level 2 exercises (learnrs, B \& C) already
  represent a larger difficulty: the number of trials per exercises is
  high (B), but not as high as those in cell 1 because they are more
  rapidly discouradged and they rely the most to the hints (C). They do
  not ask many questions (E, F, G) and contribute midly to projects (H),
  although a little bit more to group projects (I). They are amongst the
  latests students to complete the exercises (A).
\end{itemize}

To summarize, we observe a clear change in the habits from one course to
the other from cell 1 to 4 to 7. The other cell emphasize different
problems. Cells 5 and 8 correspond to students that have to contribute
unusually highly to group projects. Cells 6 and 9 contain students that
have difficulties to communicate, and do so quasi exclusively by email
(6) or not at all (9). Cell 2 contains students that already have more
difficulties than the orthers at the level 2 of difficulty (learnrs) and
that are discouradged faster (and look at the hints instead of trying
harder to figure out the answer by themselves). finally, cell 3 contain
students that fail and do not complete the exercises.

\hypertarget{transition-between-face-to-face-and-distance-learning}{%
\subsection{Transition between face-to-face and distance
learning}\label{transition-between-face-to-face-and-distance-learning}}

Due to Covid-19 lockdown periods, distance learning had to be adopted
abruptly. We analyse the activity and support data collected during
academic years 2019-2020 and 2020-2021 to assess the impact of this
transition on the progression of the students. The academic year is
divided into work periods of approximately 2 weeks. The courses of the
second term of the year 2019-2020 start in period Y1P09 and the courses
of the first term of the year 2020-2020 start in period Y2P01. The first
lockdown started in period Y1P11 and the second in Y2P03. During the
first lockdown, support rapidly switched to the proposed channels, by
email and via Discord.

\begin{figure}
\includegraphics[width=5.97in]{figures/add_per_message} \caption{Support productivity as the number of lines added in the project reports divided by the number of message sent via Discord of by email (log scale). The total number of students that sent one or more messages during each period is indicated on top of the boxplots.}\label{fig:timing}
\end{figure}

The activity in the reports remains relatively proportional to the
number of questions the students sent by email or Discord messages, no
matter the period and the intensity of the work as indicated by the
number of modules to be completed during the period, all three courses
pooled together. Only the number of students that ask questions by there
channels change between face-to-face and distance teaching (much less in
face-to-face because most of the students ask their questions directly
in the classroom). Transition from direct interaction to electronic
exchange was quasi immediate during lockdown. Consequently, support
provided by the teachers in distance learning \ldots{} to be continued.

\hypertarget{discussion}{%
\section{Discussion}\label{discussion}}

{[}juste quelques idées\ldots{} à développer et à traduire en anglais
bien sûr.{]}

\begin{itemize}
\item
  L'examen en fin de période, même s'il reprend des questions liées à de
  la pratique et de l'utilisation d'outils, ne mène pas à une évaluation
  formtement corrélée avec l'activité qui nous intéresse le plus, à
  savoir, la capacité de l'étudiant à analyser des données billogiques
  réelles. Cette capacité est parfaitement évaluée dans les projets de
  niveau 1, et surtout de niveau 2 qui correspondent très précisément à
  une telle activité. Par conséquent, l'évaluation ne se fait plus via
  un examen final, mais uniquement via les prestations des étudiants
  dans les projets, ainsi que (pour une part relativement faible de 15\%
  de la note finale), leur progression dans l'apprentissage de la
  matière via la réalisation des exercises de niveau 1 et des tutoriels
  de niveau 2, ceci afin de les encourager à réaliser complètement tous
  les exercises et à les faire dans l'ordre croissant de difficulté.
\item
  Même au sein d'une cohorte d'étudiants ayant un parcours académique
  similaire, nous notons de très grosses différences de stratégie dans
  les activités d'apprentissage. Si plusieurs stratégies différentes
  sont associées à une acquisition bonne à erxcellente des compétences
  telle qu'attestée par les notes obtenues, plusieurs profils sont
  systématiquement associés à des performances faibles. Les profils
  ainsi établis via cartes auto-adaptatives permettront à l'avenir de
  détecter plus tôt les étudiants à suivre plus particulièrement et à
  réfléchir à des approches alternatives pour eux afin de les aider
  (pédagogie inclusive).
\item
  Les études portant sur le changements d'attitudes au sein de semestre
  ne montre pas différence significative. La comparaison entre les 3
  cours met en avant qu'il faut plusieurs cours en continu afin
  d'observer une changement de la charge cognitive des étudiants.
\item
  Apprentissage en continu sur 3 années successives (cohérence entre le
  programme et l'approche pédagogique), les résultats sont meilleurs
  vers la 3ieme années.
\item
  Pendant les périodes de confinements, le passage brutal à des cours en
  présentiel vers des cours en distaznciel nécesite une période
  d'adaptation que nous avons quantifié dans notre cas à environ 2
  semaine. Il s'agit ici du temps d'adaptation des étudiants, sachant
  que du côté des enseignants, nous avons réagit immédiatement (et même
  anticipé) en mettant en place très rapidement les canaux de
  communication alternatifs via le mail et Discord.
\item
  Les tutoriels learnrs jouent un rôle charnière entre la théorie et la
  pratique. Ils offrent la possibilité de préparer les étudiants de
  manière optimale à l'analyse de données en pratique. Nous avans
  quantifié la charge cognitive perçue. Si la valeur absolue de l'index
  RTX n'est pas informative, la comparaison d'index obtenus dans des
  situations différentes permet de déterminer laquelle de ces situations
  est la mieux perçue. La compraison des trois cours successifs montre
  une diminution de cette charge cognitive perçue dans le dernier cours
  qui est pourtant le plus avancé et le plus difficile. A l'avenir, nous
  pourrons utiliser ces points de référence pour encore améliorer ces
  tutoriels de ce pôint de vue.
\end{itemize}

\hypertarget{conclusions}{%
\section{Conclusions}\label{conclusions}}

\begin{itemize}
\item
  Exam classique évalue mal la capacité d'evaluer des données
  biologiques par eux même
\item
  Les biologistes non expert de l'informatique est une challenge vu le
  nombre important de notions a apprendre utilisation d'un ordi, gestion
  de projet, statistique. Il faut décomposer ces notions en petites
  étapes successives si nous ne voulons pas les perdre rapidement. Notre
  approche en 3 cours étalés sur 5 quadrimestres successifs et étalés
  sur 3 années semblent correspondre à un bon timing pour ce type
  d'étudiant qui, au départ, n'a aucune notion de statistique, et très
  peu de connaissance des outils des logiciels couramment utilisées par
  le scientifique des données.
\item
  Néanmoins, malgré leur habituation progressive, ces logiciels restent
  vus comme pointu et diffficile d'utilisation (SUS) {[}à voir si on met
  cela dans l'article: on a déjà beaucoup ! =\textgreater{} réserver
  cela pour un autre article l'annéde prochaine peut-être ?{]}.
\item
  l'evaluation continue et l'analyse de projet via des grilles critérié
  semble une approche intéressante pour juger de la capacité des
  étudiant à bosser {[}on a pas développé cela au final, il me
  semble{]}.
\item
  la catégorisation des étudiants en différents profils d'apprenants
  ayant adopté des stratégies très contrastées démontre une grande
  diversité des apprenants, même à l'intérieur d'un groupe a priori
  homogène (n,ous ne nous trouvons pas ici dans une grande classe qui
  regrouperait des étudiants d'horizons très différents comme les cours
  d'introduction à la science des données tels que pratiques dans
  certaines grandes universités américaines). Ceci est un premier pas
  vers une pédagogie différencié et plus inclusive qui s'avèrent être
  des éléments importants ici.
\end{itemize}

\hypertarget{references}{%
\section*{References}\label{references}}
\addcontentsline{toc}{section}{References}

\hypertarget{refs}{}
\begin{CSLReferences}{1}{0}
\leavevmode\hypertarget{ref-Auker2020}{}%
Auker, Linda A., and Erika L. Barthelmess. 2020. {``{Teaching R in the
undergraduate ecology classroom: approaches, lessons learned, and
recommendations}.''} \emph{Ecosphere} 11 (4): e03060.
https://doi.org/\url{https://doi.org/10.1002/ecs2.3060}.

\leavevmode\hypertarget{ref-Baker2016}{}%
Baker, Monya. 2016. {``1,500 Scientists Lift the Lid on
Reproducibility.''} \emph{Nature} 533 (7604): 452--54.
\url{https://doi.org/10.1038/533452a}.

\leavevmode\hypertarget{ref-Banks2019}{}%
Banks, George C., James G. Field, Frederick L. Oswald, Ernest H.
O'Boyle, Ronald S. Landis, Deborah E. Rupp, and Steven G. Rogelberg.
2019. {``{Answers to 18 Questions About Open Science Practices}.''}
\emph{Journal of Business and Psychology} 34 (3): 257--70.
\url{https://doi.org/10.1007/s10869-018-9547-8}.

\leavevmode\hypertarget{ref-Baumer2014}{}%
Baumer, Ben, Mine Cetinkaya-Rundel, Andrew Bray, Linda Loi, and Nicholas
J. Horton. 2014. {``{R Markdown: Integrating A Reproducible Analysis
Tool into Introductory Statistics}.''} \emph{Technology Innovations in
Statistics Education} 8 (1). \url{https://doi.org/10.5070/t581020118}.

\leavevmode\hypertarget{ref-Boettiger2015}{}%
Boettiger, Carl. 2015. {``{An Introduction to Docker for Reproducible
Research}.''} \emph{SIGOPS Oper. Syst. Rev.} 49 (1): 71--79.
\url{https://doi.org/10.1145/2723872.2723882}.

\leavevmode\hypertarget{ref-Byers1989}{}%
Byers, J C, A Bittner, and S Hill. 1989. {``{Traditional and raw task
load index (TLX) correlations: Are paired comparisons necessary? In
A}.''} In.

\leavevmode\hypertarget{ref-Cetinkaya-Rundel2021}{}%
Çetinkaya-Rundel, Mine, and Victoria Ellison. 2021. {``{A Fresh Look at
Introductory Data Science}.''} \emph{Journal of Statistics Education} 0
(0): 1--27. \url{https://doi.org/10.1080/10691898.2020.1804497}.

\leavevmode\hypertarget{ref-Cetinkaya-Rundel2018}{}%
Çetinkaya-Rundel, Mine, and Colin Rundel. 2018. {``{Infrastructure and
Tools for Teaching Computing Throughout the Statistical Curriculum}.''}
\emph{American Statistician} 72 (1): 58--65.
\url{https://doi.org/10.1080/00031305.2017.1397549}.

\leavevmode\hypertarget{ref-Donoho2017}{}%
Donoho, David. 2017. {``{50 Years of Data Science}.''} \emph{Journal of
Computational and Graphical Statistics} 26 (4): 745--66.
\url{https://doi.org/10.1080/10618600.2017.1384734}.

\leavevmode\hypertarget{ref-Estrellado2020}{}%
Estrellado, Ryan A., Emily A. Bovee, Jesse Mostipak, Joshua M.
Rosenberg, and Isabella C. Velásquez. 2020. \emph{{Data science in
education using R}}. London, England: Routledge.
\url{https://datascienceineducation.com/}.

\leavevmode\hypertarget{ref-Fiksel2019}{}%
Fiksel, Jacob, Leah R. Jager, Johannna S. Johanna S Hardin, and Margaret
A. Taub. 2019. {``{Using GitHub Classroom To Teach Statistics}.''}
\emph{Journal of Statistics Education} 27 (2): 110--19.
\url{https://doi.org/10.1080/10691898.2019.1617089}.

\leavevmode\hypertarget{ref-Freeman2014}{}%
Freeman, Scott, Sarah L. Eddy, Miles McDonough, Michelle K. Smith,
Nnadozie Okoroafor, Hannah Jordt, and Mary Pat Wenderoth. 2014.
{``{Active learning increases student performance in science,
engineering, and mathematics}.''} \emph{Proceedings of the National
Academy of Sciences} 111 (23): 8410--15.
\url{https://doi.org/10.1073/PNAS.1319030111}.

\leavevmode\hypertarget{ref-Hart1988}{}%
Hart, Sandra G., and Lowell E. Staveland. 1988. {``{Development of
NASA-TLX (Task Load Index): Results of Empirical and Theoretical
Research}.''} \emph{Advances in Psychology} 52 (C): 139--83.
\url{https://doi.org/10.1016/S0166-4115(08)62386-9}.

\leavevmode\hypertarget{ref-Hsing2019}{}%
Hsing, Courtney, and Vanessa Gennarelli. 2019. {``{Using GitHub in the
Classroom Predicts Student Learning Outcomes and Classroom Experiences:
Findings from a Survey of Students and Teachers}.''} In
\emph{Proceedings of the 50th ACM Technical Symposium on Computer
Science Education}, 672--78. SIGCSE '19. New York, NY, USA: Association
for Computing Machinery. \url{https://doi.org/10.1145/3287324.3287460}.

\leavevmode\hypertarget{ref-Krathwohl2002}{}%
Krathwohl, David R. 2002. {``{A Revision of Bloom' s Taxonomy: An
otherview}.''} \emph{Theory Into Practice} 41 (4): 212--18.
\url{https://doi.org/10.1207/s15430421tip4104}.

\leavevmode\hypertarget{ref-Larwin2011}{}%
Larwin, Karen, and David Larwin. 2011. {``{A meta-analysis examining the
impact of computer-assisted instruction on postsecondary statistics
education: 40 years of research}.''} \emph{Journal of Research on
Technology in Education} 43 (3): 253--78.
\url{https://doi.org/10.1080/15391523.2011.10782572}.

\leavevmode\hypertarget{ref-Marx2013}{}%
Marx, Vivien. 2013. {``{The big challenges of big data}.''}
\emph{Nature} 498 (7453): 255--60.
\url{https://doi.org/10.1038/498255a}.

\leavevmode\hypertarget{ref-Onwuegbuzie2003}{}%
Onwuegbuzie, Anthony J., and Vicki A. Wilson. 2003. {``{Statistics
Anxiety: Nature, etiology, antecedents, effects, and treatments--a
comprehensive review of the literature}.''} \emph{Teaching in Higher
Education} 8 (2): 195--209.
\url{https://doi.org/10.1080/1356251032000052447}.

\end{CSLReferences}

\end{document}
